%&custom12pt
%\begin{document}
\begin{table}[H]
\caption{}
\medskip{\centerline{Household Inflation Expectations Are More Accurate When There Is More News Coverage}}\medskip
\centerline{Equation Estimated: GAPSQ$_{t} = \alpha_{0} + \alpha_{1} $NEWS$_{t}$}
\medskip\medskip\medskip
\begin{center}
\begin{tabular}
{c                            d{4}                                 d{4}                                d{4}                            d{4}           } \hline\hline
\multicolumn{1}{c}{ Sample} & \multicolumn{1}{c}{$\alpha_{0}$}   & \multicolumn{1}{c}{$\alpha_{1}$} &  \multicolumn{1}{c}{D-W Stat}  & \multicolumn{1}{c}{$\bar{R}^{2}$} \\ \hline
 1981q3-2000q2   &  0.94  &  -1.03  &  1.01  &  0.08  \\
 & ( 0.26 )^{***} & ( 0.50 )^{**}  & & \\
 1982q3-2000q2   &  1.22  &  -1.72  &  1.08  &  0.14  \\
 & ( 0.25 )^{***} & ( 0.46 )^{***}  & & \\
\hline\hline \end{tabular}
\end{center}
{\footnotesize
GAPSQ is the square of the difference between the Michigan and SPF inflation forecasts.
NEWS is an index of the intensity of news coverage of inflation in the New York Times and the Washington Post from 1981 to 2000.
All standard errors are corrected for heteroskedasticity and serial correlation using a Newey-West~\cite{newey&west:hac} procedure with four lags.
Results are not sensitive to the choice of lags.  \{***,**,*\} = \{1 percent, 5 percent, 10 percent\} significance.
}
\label{table:erronnews}
\end{table}
%\end{document}
