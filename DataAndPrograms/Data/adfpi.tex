%&custom12pt
%\begin{document}
\begin{table}[h]
\begin{center}
\begin{tabular}
{|c|                            d{7}|                                 d{8}|                                } \hline
\multicolumn{1}{|c|}{ Lags } & \multicolumn{1}{c|}{Degrees of Freedom}   & \multicolumn{1}{c|}{ADF Test Statistic} \\ \hline
 0   & 166  &  3.59^{***}  \\
 1   & 165  &  2.84^{*}    \\
 2   & 164  &  2.28        \\ \hline
\end{tabular}
\end{center}

\medskip
{\footnotesize

This table presents results of standard Dickey-Fuller and Augmented 
Dickey-Fuller tests for the presence of a unit root in the core rate 
of inflation (results are similar for CPI inflation).  The column 
labelled `Lags' indicates how many lags of the change in the inflation 
rate are included in the regression.  With zero lags, the test is the 
original Dickey-Fuller test; with multiple lags, the test is an 
Augmented Dickey Fuller test.  In both cases a constant term is 
permitted in the regression equation.  The sample is from 1959q3 to 
2001q2 (quarterly data from my DRI database begin in 1959q1.  In order 
to have the same sample for all three tests, the sample must be 
restricted to 1959q3 and after.)  One, two, and three stars indicate 
rejections of a unit root at the 10 percent, 5 percent, and one 
percent thresholds.  RATS code generating these and all other 
empirical results is available at the author's website.

}
\caption{Dickey-Fuller and Augmented Dickey-Fuller Tests for a Unit Root in Inflation}
\label{table:ADFpi}
\end{table}
%\end{document}
