%&LaTeX
\documentclass[titlepage,12pt,letterpaper,leqno]{MacPocket:Software:latex:articleqje}
\usepackage{MacPocket:Software:latex:floatqje,dcolumn,MacPocket:Software:latex:vmargin,ifthen,MacPocket:Software:latex:harvard_qje,amsmath,epsfig}%
%\pagestyle{empty}
%&custom12pt

\newcolumntype{d}[1]{D{.}{.}{#1}}
\setmarginsrb{1.2in}{1in}{1.2in}{1.4in}{0pt}{0pt}{0pt}{36pt}
% {left}{top}{right}{bottom}{headheight}{headsep}{footheight}{footskip}

%&custom12pt

\newboolean{FigsAndTablesInText}
\setboolean{FigsAndTablesInText}{false}


\newboolean{QJETypesetVersion}
\setboolean{QJETypesetVersion}{false}

\ifthenelse{\boolean{QJETypesetVersion}}{
\usepackage{MacPocket:Software:latex:Endnotes_doublespace}

% The QJE wants endnotes rather than footnotes, so
% translate footnote commands to endnote commands
\let\footnote=\endnote
}{}


\begin{document}

\begin{titlepage}

\hfill{\tiny EpidemiologyQJE.tex}



\def\thefootnote{\fnsymbol{footnote}}

%\baselineskip=.2in

\enlargethispage{1000pt}
%\vspace{0.1in}

\centerline{\Large {\sc {Macroeconomic Expectations of}}}
\medskip\centerline{\Large {\sc Households and Professional 
Forecasters}$^{*}$}
\vspace{0.2in}

\centerline{\large Christopher D. Carroll}
\centerline{ccarroll@jhu.edu}

\medskip\medskip\centerline{\it Published in the Quarterly Journal of 
Economics, Volume 118, Number 1, February 2003}


\vspace{0.1in}

\centerline{\today}

\vspace{0.1in}
\begin{table}[H]

\vspace{.1in}
\centerline{\sc Abstract} 

Economists have long emphasized the importance of expectations in
determining macroeconomic outcomes.  Yet there has been almost no
recent effort to model actual empirical expectations data; instead,
macroeconomists usually simply assume that expectations are
`rational.'  This paper shows that while empirical household
expectations are not rational in the usual sense, expectational
dynamics are well captured by a model in which households' views
derive from news reports of the views of professional forecasters,
which in turn may be rational.  The model's estimates imply that
people only occasionally pay attention to news reports; this
inattention generates `stickyness' in aggregate expectations, with
important macroeconomic consequences.


\medskip \medskip {\bf Keywords}: inflation, expectations, unemployment,
monetary policy

\medskip
\medskip
{\bf JEL Classification Codes: D84, E31} 

\medskip
\medskip


\small This paper is derived from a paper originally titled ``The
Epidemiology of Macroeconomic Expectations'' written in connection
with the conference ``The Economy As an Evolving Complex System III''
at the Santa Fe Institute in November 2001, honoring Kenneth
Arrow's contributions to the Santa Fe Institute and to economics.  A
paper with the original title will be published in the associated
eponymous conference volume.  That companion paper examines a variety
of alternative epidemiological models of expectations transmission
that generate results similar to those of the baseline model presented
here.

\medskip\medskip \input titlepage_thanks.tex

\medskip\medskip

The data and econometric programs that generated all of the results in 
this paper are available on the author's website, 
http://www.econ.jhu.edu/people/carroll.

\medskip\medskip\medskip

$^{*}$  Department of Economics, The Johns Hopkins University, Baltimore MD 21218-2685.

\end{table}

\normalsize


\clearpage\pagebreak\vfill\eject

\end{titlepage}

\sloppy 
%\renewcommand{\hbadness}{10000}


\ifthenelse{\boolean{QJETypesetVersion}}{

\renewcommand{\baselinestretch}{1.5} \normalsize

}{}

\section{Introduction}

%\large \setmarginsrb{0.8in}{0.5in}{0.5in}{0.5in}{0pt}{0pt}{0pt}{0pt}% {left}{top}{right}{bottom}{headheight}{headsep}{footheight}{footskip}\pagestyle{empty}

% Macroeconomists agree that expectations affect outcomes
Ever since the traditional foundation of macroeconomics by John
Maynard Keynes~\cite{keynes:generaltheory}, economists have understood
that macroeconomic outcomes depend upon expectations.  Keynes himself
believed that economies could experience fluctuations that reflected
movements in `animal spirits,' but the basis for most of today's macro
models was laid in the rational expectations revolution of the 1970s. 
Early critics of the rational expectations approach complained that,
in the words of Friedman~\cite{friedman:extreme}, such models lacked
``a clear outline of the way in which economic agents derive the
knowledge which they then use to formulate expectations,'' but 
recent criticisms have focused on the difficulty rational
expectations models have in reproducing various features of
macroeconomic data like the high persistence of inflation (Fuhrer and
Moore~\cite{fuhrer&moore:persistence}) and the apparent inexorability
of the tradeoff between inflation and unemployment
(Ball~\cite{ball:sacrifice}; Mankiw~\cite{mankiw:inexorable}).  The
literature has consequently begun to explore the implications for
macroeconomic dynamics of various alternative assumptions about
expectations formation, most notably models of learning, see
Sargent~\cite{sargent:bounded} or Evans and
Honkapohja~\cite{evans&honk:book} for surveys.

Remarkably, however, there has been almost no work testing alternative
models of expectations using actual empirical data on expectations. 
McCallum's~\cite{mccallum:review} recent survey, for example, does not
discuss results from a single paper that examines empirical
expectations data.  This is not for lack of data: The University of
Michigan's Survey Research Center has been collecting information on
households' expectations about inflation, unemployment, economic
growth, interest rates, and other macroeconomic matters for almost 50
years, the Conference Board has conducted similar monthly surveys of
households since the late 1970s, and the Survey of Professional
Forecasters and its antecedents have collected data since the 1960s. 
While there has been some work testing (and usually rejecting) the
rationality of these expectations,\footnote{See
Croushore~\cite{croushore:spfisgood}, Thomas~\cite{thomas:surveyinfl}, 
and Mehra~\cite{mehra:surveyinfl} for surveys.} aside from an
insightful paper by Roberts~\cite{roberts:inflexp} and an impressive
(and very recent) paper by Branch~\cite{branch:hetero} there appears
to have been essentially no work proposing and testing positive
alternative models for how empirical expectations are
formed.\footnote{The only other even tangentially relevant papers I
have found were by Fishe and Idson~\cite{fishe&idson:infohetero} who
test a model of heterogeneous demand for information using two years'
worth of Michigan Survey data; a paper by Urich and
Wachtel~\cite{urich&wachtel:money} that tests rationality using survey
data on money supply forecasts; and a paper by Dua and
Ray~\cite{dua&ray:arima} that models SPF data using an ARIMA
forecasting framework.}

% This paper proposes a positive model
This paper proposes and tests one such model.  Rather than having full
understanding of the `true' macroeconomic model and constantly
tracking the latest statistics to produce their own macroeconomic
forecasts, typical people are assumed to obtain their macroeconomic
views from the news media.  Furthermore (and importantly), not every
person pays close attention to all macroeconomic news; instead,
individual people are assumed to absorb the economic content of news
stories probabilistically, so that it may take quite some time for
news of changed macroeconomic circumstances to penetrate to all agents
in the economy.  Finally, the news media in turn are assumed to report
the views of professional forecasters, who may themselves make
rational forecasts.  (The structure of the model was inspired by
simple models of disease from the epidemiology literature; see the
companion paper Carroll~\cite{carroll:epidemicinflSFI}) for more on
the epidemiological foundations of the model and for a demonstration
that more elaborate ``epidemiological expectations'' models generate
results similar to those of the baseline model presented here.)

% Provides simple equation similar to Mankiw and Reis; microfoundations
The baseline model provides a simple equation for the evolution of
mean expectations that is very similar to an equation proposed in
recent papers by Mankiw and
Reis~\cite{mankiw&reis:slumps,mankiw&reis:stickye}.  Indeed, the model
presented here could be viewed as providing microfoundations for the
Mankiw and Reis equation.  Another contribution is the derivation of a
particularly simple specialization of that equation suitable for
empirical work; this specialization turns out to yield an equation
like one estimated by Roberts~\cite{roberts:inflexp}, for which it can
again be regarded as providing a microfoundation.  Finally, the
model's explicit assumption that people derive their expectations from
news reports (and the paper's specific proposal for how to measure
news coverage) respond to Friedman's~\cite{friedman:extreme}
criticism of the unspecified nature of the expectations formation
mechanism in rational expectations models.

% Application to inflation expectations: once a year
The model is applied to estimate the evolution of inflation
expectations and unemployment expectations from the Michigan Survey of
Consumers.  For inflation, the typical household is estimated to
update expectations roughly once a year, while unemployment
expectations appear to be updated slightly more frequently. 
Furthermore, in a horserace between a version of the model where
people update their expectations either to the rational
forward-looking forecast or to the most recently reported past
statistics (the `adaptive expectations' model), the data strongly
prefer the forward-looking version of the model.  Thus, the results
can be interpreted as reflecting a plausible middle ground between
fully rational expectations and adaptive expectations.

% Implications for macro dynamics not addressed here; see M&R and Roberts
A final section briefly comments on the relationship between this
model and some of the relevant existing empirical literature, with
particular emphasis on the relationship of the model to sticky-price
models and the model's implications for the relationship between
credibility and monetary policy.  The implications of the model for
macroeconomic dynamics are not addressed here, because the papers by
Mankiw and Reis~\cite{mankiw&reis:slumps,mankiw&reis:stickye} and
Roberts~\cite{roberts:phillips,roberts:stickyinfl} address those
questions and are directly applicable.  Mankiw and Reis show that
their model can explain many phenomena that are unexplained by fully
rational models, including why disinflations are inevitably
contractionary; why monetary policy affects the economy with
considerable lags; why rapid economic growth leads to rising
inflation; and why productivity slowdowns are associated with a rise
in the natural rate of unemployment.  The ability to solve all of
these puzzles seems a large dividend in exchange for the small price
of relaxing the assumption that all agents' expectations are fully
rational (in the sense required by typical rational expectations
models) at every instant.

% Relation to literature on credibility

\section{The Model}


% Suppose people form inflation expectations by reading newspapers
Consider a world where most people form their expectations about
future inflation by reading newspaper articles.  Imagine for the
moment that every inflation article contains a complete forecast of
the inflation rate for all future quarters, and suppose (again
momentarily) that any person who reads such an article can
subsequently recall the entire forecast.

% Not everybody reads every article; probabilistic
Assume that not everybody reads every newspaper article on inflation. 
Instead, in any given period each individual faces a constant
probability $\lambda$ of encountering and absorbing the contents of an
article on inflation.  Individuals who do not encounter an article
simply continue to believe the last forecast they read about.  Thus,
the framework is mathematically similar to the
Calvo~\cite{calvoPrices} model of sticky prices in which firms change
their prices with probability $p$.

% Leads to expectations being infinite distributed lag
Call $\pi_{t+1}$ the inflation rate between quarter $t$ and quarter $t+1$,
\begin{eqnarray*}
 	\pi_{t+1} & = & \log(p_{t+1})-\log(p_{t}),
\end{eqnarray*}
where $p_{t}$ is the aggregate price index in period $t$.  If we 
define $M_{t}$ as the operator that yields the population-mean value 
of inflation expectations at time $t$ and denote the {\bf N}ewspaper 
forecast printed in quarter $t$ for inflation in quarter $s \geq t$ as 
$N_{t}[\pi_{s}]$, we have that\footnote{Here we are assuming that all 
newspapers report the same forecast for inflation; see 
Carroll~\cite{carroll:epidemicinflSFI} for a version that allows for 
the possibility that different newspapers print different forecasts; 
that paper shows that results in such a model are similar to those 
presented here.}
\begin{eqnarray}
	M_{t}[\pi_{t+1}] & = & \lambda N_{t}[\pi_{t+1}]
	+(1-\lambda) \left\{ \lambda N_{t-1}[\pi_{t+1}]
	+(1-\lambda)\left(\lambda N_{t-2}[\pi_{t+1}] 
	+ \ldots \right) \right\}. \label{eq:stickyerat}
\end{eqnarray}

The derivation of this equation is as follows.  In period $t$ a 
fraction $\lambda$ of the population will have absorbed the 
current-period newspaper forecast for the next quarter, 
$N_{t}[\pi_{t+1}]$.  Fraction $(1-\lambda)$ of the population retains 
the views they held in period $t-1$ of period $t+1$'s inflation rate.  
Those period-$t-1$ views in turn can be decomposed into a fraction 
$\lambda$ of people who encountered an article in period $t-1$ and 
obtained the newspaper forecast of period $t+1$'s forecast, 
$N_{t-1}[\pi_{t+1}]$, and a fraction $(1-\lambda)$ who retained their 
period-$t-2$ views about the inflation forecast in period $t+1$.  
Recursion leads to the remainder of the equation.

% Similar to M&R expression
This expression for inflation expectations is identical to the one
proposed by Mankiw and
Reis~\cite{mankiw&reis:slumps,mankiw&reis:stickye}, except that in
their framework updating agents compute their own forecasts under the
usual assumptions of rational expectations.  Mankiw and Reis motivate
their assumption that forecasts are updated only occasionally by
arguing that there may be costs to obtaining or processing
information.  It is undoubtedly true that developing a reasonably
rational quarter-by-quarter forecast of inflation arbitrarily far into
the future would be a very costly enterprise for a typical person (for
example, it might require obtaining an economics Ph.D. first!).  But
Mankiw and Reis do not provide any formal model of information
processing costs that leads to their specification, and indeed it
seems likely that a formal model of processing costs might imply an
updating process quite different from the Poisson process Mankiw and
Reis assume.\footnote{See Sims~\cite{sims:inattention} for a
model grounded in information theory that provides a formal model of
decisionmaking under information-processing constraints.}

% epidemiological model is a microfoundation
The model proposed above can be regarded as a microfoundation for the
Mankiw and Reis equation~\eqref{eq:stickyerat}.  Its value as a
microfoundation is illustrated in the usual way: it provides
additional testable implications that do not follow directly from the
aggregate specification.  In particular, the baseline model implies
that in periods when there are more news stories on inflation, the
speed of updating should be faster, an implication that is borne out
in the empirical work below.  It also provides implications for the
analysis of the underlying micro data from the Michigan survey.  In
particular, Souleles~\cite{souleles:sentiment} finds highly
statistically significant differences across demographic groups in
macroeconomic forecasts; the model suggests examining
whether those differences can be explained by information on
demographic differences in readership rates of newspapers, or more
general data on differences in the extent to which different groups
pay attention to economic matters.

% Need to relax assumption that newspapers print infinite forecasts
Of course, real newspaper articles do not contain a quarter-by-quarter
forecast of the inflation rate into the infinite future as assumed in
the derivation of \eqref{eq:stickyerat}, and even if they did it is
very unlikely that a typical person would be able to remember the
detailed pattern of inflation rates far into the future.  Furthermore,
even if both of these assumptions were true,
equation~\eqref{eq:stickyerat} wound not be testable in its current
form because the available survey data report only households'
expectations about inflation rates over the next year.  In order to
derive implications from the model that are testable with these data,
it turns out to be necessary to impose some structure on households'
implicit views about the inflation process.

% Suppose people believe in a fundamental inflation rate
Suppose people believe that at any given time the economy has an 
underlying ``fundamental'' inflation rate.  Furthermore, suppose they 
believe that future changes in the fundamental rate are 
unforecastable; that is, beyond the next period the fundamental rate 
follows a random walk.  Finally, assume that people believe that the 
actual inflation rate in a given quarter is equal to that period's 
fundamental rate plus an error term $\epsilon_{t}$ which reflects 
unforecastable transitory inflation shocks (reflected in the `special 
factors' that newspaper inflation stories often emphasize).  Thus, the 
typical person believes that the inflation process is captured by
\begin{eqnarray}
	\pi_{t} & = & \pi_{t}^{f}+\epsilon_{t}            \label{eq:Fpi} 
\\   \pi_{t+1}^{f} & = & \pi_{t}^{f}+\eta_{t+1}  \label{eq:Fpi2},
%\\   \pi_{t+2}^{f} & = & \pi_{t+1}^{f}+\eta_{t+2}  \label{eq:Fpi3}
\\ \phantom{456}\vdots\phantom{3} & & 
\phantom{\ldots .}\vdots\phantom{\ldots} \nonumber
\end{eqnarray}
where $\epsilon_{t}$ is a transitory shock to the inflation rate in
period $t$ while $\eta_{t}$ is the permanent innovation in the
fundamental inflation rate $\pi^{f}_{t}$ in period $t$.  Now assume
that consumers believe that values of $\eta$ beyond period $t+1$, and
values of $\epsilon$ beyond period $t$, are unforecastable white noise
variables; that is, future changes in the fundamental inflation rate
are unforecastable, and transitory shocks are expected to go
away.\footnote{Note that we are allowing people to have some idea
about how next quarter's fundamental rate may differ from the current
quarter's fundamental rate, because we did not impose that consumers'
expectations of $\eta_{t+1}$ must equal zero.}

% Is this plausible?  Cite the literature
Before proceeding it is worth considering whether this is a plausible
view of the inflation process; we would not want to assume that people
believe something patently absurd.  However, the near-unit-root
feature of the inflation rate in the post-1959 period is well known to
inflation researchers; some authors find that a unit root can be
rejected for some measures of inflation over some time periods, but it
seems fair to say that the conventional wisdom is that at least since
the late 1950s inflation is `close' to a unit root process.  See
Barsky~\cite{barsky:fisher} for a more complete analysis, or
Ball~\cite{ball:nearrational} for a more recent treatment.

% Not in conflict with the Phillips curve
Note that the unit root (or near unit root) in inflation does not 
imply that future inflation rates are totally unpredictable, only that 
the history of inflation by itself is not very useful in forecasting 
future inflation {\it changes} (beyond the disappearance of the 
transitory component of the current period's shock).  This does not 
exclude the possibility that current and lagged values of other 
variables might have predictive power.  Thus, this view of the 
inflation rate is not necessarily in conflict with the vast and 
venerable literature showing that other variables (most notably the 
unemployment rate) do have considerable predictive power for the 
inflation rate (see Staiger, Stock, and Watson~\cite{ssw:nairu} for a 
recent treatment).

% An ARIMA forecast is not optimal if the experts know something

If we were to assume that households were rational and made their own
inflation forecasts solely based on observed past inflation under the
assumption of an inflation process like
\eqref{eq:Fpi}-\eqref{eq:Fpi2}, then the rational forecast would be a
geometrically declining weighted average of past inflation
realizations; in this case rational expectations would be identical to
adaptive expectations (Muth~\cite{muthOptimal}).  However, we will
assume that households believe that {\it experts} have some ability to
directly estimate the past and present values of $\epsilon$ through
period $t$ and $\eta$ through period $t+1$ (through deeper knowledge
of how the economy works, or perhaps some private information); thus
households can rationally believe that a forecast from a professional
forecaster is more accurate than a simple adaptively rational forecast
that they could construct themselves.

% Newspapers print forecast for next year; integrate into model
Suppose now that rather than containing a forecast for the entire 
quarter-by-quarter future history of the inflation rate, newspaper 
articles simply contain a forecast of the inflation rate over the next 
year.  The next step is to figure out how such a one-year forecast for 
inflation can be integrated into some modified version of 
equation~\eqref{eq:stickyerat}.  To capture this, we must introduce a 
bit more notation.  Define $\pi_{s,t}$ as the inflation rate between 
periods $s$ and $t$, converted to an annual rate.  Thus, for example, 
in quarterly data we can define the inflation rate for quarter $t+1$ 
at an annual rate as
\begin{eqnarray*}
	\pi_{t,t+1} & = & 4 (\log p_{t+1} - \log p_{t})
\\  & = & 4 \pi_{t+1}	,
\end{eqnarray*}
where the factor of four is required to convert the quarterly price 
change to an annual rate.

% What would people expect ex-post inflation rate to be?
Our hypothetical person's view is that the true {\it ex-post} 
inflation rate over the next year will be given by
\begin{eqnarray}
    \pi_{t,t+4} & = & \pi_{t+1}+\pi_{t+2}+\pi_{t+3}+\pi_{t+4}  \label{eq:pittp4}
\\   & = & \pi_{t+1}^{f}+\epsilon_{t+1}+\pi_{t+2}^{f}+\epsilon_{t+2}+\pi_{t+3}^{f}+\epsilon_{t+3}+\pi_{t+4}^{f}+\epsilon_{t+4} \nonumber 
\\   & = & \pi_{t+1}^{f}+\epsilon_{t+1}+\pi_{t+1}^{f}+\eta_{t+2}+\epsilon_{t+2}+\pi_{t+1}^{f}+\eta_{t+2}+\eta_{t+3}+\epsilon_{t+3}+ \nonumber
\\   & & \pi_{t+1}^{f}+\eta_{t+2}+\eta_{t+3}+\eta_{t+4}+\epsilon_{t+4}.  \nonumber 
\end{eqnarray}

% Define forecast of an updating agent as F[\bullet]
Define $F_{t}[\bullet_{s}]$ as the agent's forecast (expectation) as 
of date $t$ of $\bullet_{s}$, for an agent who updates his views from 
a news report in period $t$.  Using this notation, the assumptions 
made earlier about the stochastic processes for $\epsilon$ and $\eta$ 
imply that 
\begin{equation}
F_{t}[\epsilon_{t+n}]=F_{t}[\eta_{t+n+1}]=0 \label{eq:noise}
\end{equation}
for all $n>0$.  Applying the $F_{t}$ operator to both sides
of~\eqref{eq:pittp4} reveals that the person's forecast of the
inflation rate over the next year is simply equal to four times his
forecast of the fundamental inflation rate for next quarter:
\begin{eqnarray*}
    F_{t}[\pi_{t,t+4}] & = & 4 F_{t}[\pi_{t+1}^{f} ]
\\  & = & F_{t}[\pi_{t,t+1}^{f}]    .
%\\  & = & F_{t}[\pi_{t,t+1}]    
\end{eqnarray*}

% Thus, people should expect what is printed in the paper to be fundamental rate
Now for an important conclusion: If people believe that the forecasts
printed in the newspaper embody the same view of the inflation process
laid out in equations~\eqref{eq:Fpi}-\eqref{eq:Fpi2} and
\eqref{eq:noise}, then an identical analysis leads to the conclusion
that (defining the `newspaper expectations' operator $N_{t}$ similarly
to the consumer's expectations operator):
\begin{eqnarray*}
    N_{t}[\pi_{t,t+4}] & = & 4 N_{t}[\pi_{t+1}^{f} ]
\\  & = & N_{t}[\pi_{t,t+1}^{f}]    .
%\\  & = & N_{t}[\pi_{t,t+1}]    
\end{eqnarray*}

Thus, from the consumer's point of view the newspaper forecast
contains only a single important piece of information: a projection of
the fundamental inflation rate over the next year, which the
process~\eqref{eq:Fpi}-\eqref{eq:Fpi2} implies is the
expected fundamental rate in all of the year's constituent quarters
and all subsequent quarters as well.  A consumer who reads the
newspaper in period $t$, therefore, will update his expectations to
equal the corresponding newspaper forecasts:
\begin{eqnarray*}
    F_{t}[\pi_{t,t+1}] =  F_{t}[\pi_{t,t+4}] = F_{t}[\pi_{t,t+4}^{f}] & = & N_{t}[\pi_{t,t+4}^{f} ] = N_{t}[\pi_{t,t+4}].
%\\  F_{t}[\pi_{t+1}] & = & N_{t}[\pi_{t+1}].
\end{eqnarray*}

The rightmost equality holds because the consumer assumes that for
$n>0$, newspaper has no information about $\epsilon_{t+n}$ or
$\eta_{t+n+1}$, so $N_{t}[\epsilon_{t+n}]=N_{t}[\eta_{t+n+1}]=0$.  The
next equality to the left holds because we assume that when the
consumer reads the newspaper his views are updated to the views
printed in the newspaper.  The other two equalities similarly hold
because $F_{t}[\epsilon_{t+n}]=F_{t}[\eta_{t+n+1}]=0$.

Now note a crucial point: the assumption that changes in the inflation
rate beyond period $t+1$ are unforecastable means that 
\begin{eqnarray}
	F_{t-1}[\pi_{t-1,t+3}] & = & F_{t-1}[\pi_{t,t+4}] \label{eq:ftm1} \\
	F_{t-2}[\pi_{t-2,t+2}] & = & F_{t-2}[\pi_{t,t+4}] \label{eq:ftm2} \\ 
	\ldots & & \label{eq:ftmn} \nonumber
\end{eqnarray}

An equation similar to \eqref{eq:stickyerat} can be written for 
projections of the inflation rate over the next year:
\begin{eqnarray*}
	M_{t}[\pi_{t,t+4}] & = & \lambda F_{t}[\pi_{t,t+4}]
	+(1-\lambda) \left\{ \lambda F_{t-1}[\pi_{t,t+4}]
	+(1-\lambda)\left(\lambda F_{t-2}[\pi_{t,t+4}] 
	+ \ldots\right) \right\} \nonumber,
\end{eqnarray*}
and substituting~\eqref{eq:ftm1}-\eqref{eq:ftm2} into this equation
and replacing $F_{t}$ with $N_{t}$ on the assumption that the newspaper
forecasts are the source of updating information, we obtain
\begin{eqnarray}
	M_{t}[\pi_{t,t+4}] & = & \lambda F_{t}[\pi_{t,t+4}]
	+(1-\lambda) \left\{ \lambda F_{t-1}[\pi_{t-1,t+3}]
	+(1-\lambda)\left( %\lambda F_{t-2}[\pi_{t-2,t+2}] 
	\ldots\right) \right\} \nonumber \\ 
	M_{t}[\pi_{t,t+4}] & = & \lambda F_{t}[\pi_{t,t+4}]
	+(1-\lambda) M_{t-1}[\pi_{t-1,t+3}] \nonumber \\
	M_{t}[\pi_{t,t+4}] & = & \lambda N_{t}[\pi_{t,t+4}]
	+(1-\lambda) M_{t-1}[\pi_{t-1,t+3}]. \label{eq:esteqn}
\end{eqnarray}

% Mean inflation expectations are average of new and old views
That is, mean measured inflation expectations for the next year should 
be a weighted average between the current `rational' (or newspaper) 
forecast and last period's mean measured inflation expectations.  This 
equation is therefore directly estimable, assuming an appropriate 
proxy for newspaper expectations can be found.\footnote{This 
equation is basically the same as equation (5) in 
Roberts~\cite{roberts:inflexp}, except that Roberts proposes that the 
forecast toward which household expectations are moving is the 
`mathematically rational' forecast (and he simply proposes the 
equation without constructing a microfoundation that might produce 
it).}

% If you don't like the derivation, here's what it buys 
Readers uncomfortable with the strong assumptions needed to 
derive~\eqref{eq:esteqn} may be happier upon noting that the equation
\begin{eqnarray}
 M_{t}[\pi_{t,t+4}] & = & \lambda N_{t}[\pi_{t,t+4}] + (1-\lambda) M_{t-1}[\pi_{t,t+4}] \label{eq:pinoapprox}
\end{eqnarray}
can be derived without any assumptions on consumers' beliefs about the 
inflation process; the difference between~\eqref{eq:esteqn} and 
\eqref{eq:pinoapprox} is only in the subscript on the $\pi$ term 
inside the $M_{t-1}$ operator.  The assumptions made above were those 
necessary to rigorously obtain $M_{t-1}[\pi_{t,t+4}] = 
M_{t-1}[\pi_{t-1,t+3}]$.  In practice, however, even a much more 
realistic view of the inflation process would likely imply a very high 
degree of correlation between the period-$t-1$ projection of the 
inflation rate over the year beginning in quarter $t$ and the 
period-$t-1$ projection of the inflation rate over the year beginning 
in quarter $t+1$.  Indeed, three of out of the four quarters ($t+2, 
t+3, $ and $t+4$) are identical between the two projections; the only 
differences between the two measures would have to spring from the 
consumer's period-$t-1$ projection of the difference between the 
inflation rates in quarters $t+1$ and $t+5$.

\section{Estimation}

% Must identify data sources
Estimating equation~\eqref{eq:esteqn} requires us to identify data 
sources for population-mean inflation expectations and for `newspaper' 
forecasts of inflation over the next year.

% Michigan for HHs
The University of Michigan's Survey Research Center conducts a monthly 
survey of households that is intended to be representative of the 
population of the United States.  One component of the survey asks 
households what they expect the inflation rate to be over the next 
year.\footnote{Specifically, households are asked whether they think 
prices will go up, stay the same, or fall over the next year.  Those 
who say `go up' (the vast majority) are then asked `By about what 
percent do you expect prices to go up, on the average, during the next 
12 months?'  For more details on the survey methodology, see 
Curtin~\cite{curtin:inflsurvQJE}).} I will directly use the mean 
inflation forecast from this survey as my proxy for 
$M_{t}[\pi_{t,t+4}]$.

% SPF for N
Identifying the `newspaper' forecast for next-quarter inflation might 
seem more problematic, but there is a surprisingly good candidate: The 
mean four-quarter inflation forecast from the Survey of Professional 
Forecasters (henceforth, SPF).  The SPF, currently conducted by the 
Federal Reserve Bank of Philadelphia and previously a joint product of 
the National Bureau of Economic Research and the American Statistical 
Association, has collected and summarized forecasts from leading 
private forecasting firms since 1968.  The survey questionnaire is 
distributed once a quarter, just after the middle of the second month 
of the quarter, and responses are due within a couple of weeks.  The 
survey asks participants for quarter-by-quarter forecasts, spanning 
the current and next five quarters, for a wide variety of economic 
variables, including GDP growth, various measures of inflation 
including CPI inflation, and the unemployment rate.  For more details 
on the SPF, see Croushore~\cite{croushore:introSPF}.

% Obvious candidates for N interviews are SPF forecasters
As noted above, the typical newspaper article on inflation interviews 
some `experts' on inflation.  The obvious candidates for such experts 
are the set of people who forecast the economy for a living, so the 
pool of interviewees is likely to be approximately the same group of 
forecasters whose views are summarized by the SPF. Hence, it seems 
reasonable to identify $N_{t}$ with the SPF inflation expectations 
data.

\subsection{Do the Forecasts Forecast?}

% Existing literature is mostly on rationality 
There is a substantial existing literature on the forecasting 
performance of various measures of inflation expectations including 
the Michigan Survey, the SPF, and an informal survey of economists 
known as the Livingston survey.\footnote{Unfortunately, the recent 
survey paper by Thomas~\cite{thomas:surveyinfl} largely neglects the 
SPF, and focuses instead mainly on comparisons of the Michigan survey 
and the Livingston survey.  Thomas finds the median of the Michigan 
survey to be a better forecaster than its mean, but my model delivers 
predictions only for the mean and not for the median, so I neglect the 
median forecast in my empirical work.} Early papers 
(Turnovsky~\cite{turnovsky:inflsurv}, Bryan and 
Gavin~\cite{bryan&gavin:inflsurv}) claimed to find statistically 
significant biases in some of the survey measures, but a recent review 
by Croushore~\cite{croushore:spfisgood} shows that some of those 
results were spurious (due to improper treatment of the data or 
econometric problems), and that the results claiming to reject 
rationality of the SPF fail to hold up when the sample period is 
updated to include data for the last ten or fifteen years.  Croushore 
specifically examines the CPI forecasts of both the Michigan survey 
and the SPF, and in a `forecast improvement' exercise finds evidence 
of systematic bias in the Michigan survey but not in the SPF. 
Roberts~\cite{roberts:stickyinfl} also finds that the Michigan 
survey's inflation expectations measure fails standard rationality 
tests.

% Sniff tests
These results are suggestive, but are not precisely targeted on the 
question we are interested in: Whether the SPF forecast can be viewed 
as `more rational' than the Michigan forecast, and whether there is 
evidence that information moves from the SPF forecasters to the 
Michigan households but not vice versa.

% Simplest measure is MSE
One of the simplest measures of forecast accuracy is the mean squared
error.  It is reassuring therefore that over the time period for which
both SPF and Michigan forecasts are available, the {\it ex post} MSE
of the SPF forecast is about 0.6 while the MSE for the Michigan survey
is almost twice as large, about 1.1.  (These are calculated by taking
the square of the difference between the respective mean forecasts and
the actual CPI core inflation over the corresponding time period.)

% Do they have ANY forecasting power? Yes
A natural next question is whether each of the two surveys has 
meaningful forecasting power for future inflation, and if so, whether 
the SPF forecast is better.  As a first step, consider the 
implications of the near unit root in inflation.  High serial 
correlation means that future levels of the inflation rate will be 
highly predictable based on the recent past history of inflation.  
Hence it is not very impressive to find that both surveys have highly 
significant predictive power for inflation (which they do), since this 
result could hold even if the forecasts were both mindless 
extrapolations of past inflation into the future.  The interesting 
question is whether the survey forecasts have predictive power for the 
future inflation rate {\it beyond} what could be predicted based on 
past inflation data.

\ifthenelse{\boolean{FigsAndTablesInText}}{\input :DataAndPrograms:Data:piforc.tex}{}

% Pi on E[Pi] and Pi lagged
To answer this question, Table~\ref{table:piforc} presents a 
regression of the actual inflation rate over the next year on the 
Michigan and SPF measures of expected inflation, along with the most 
recent annual inflation statistic available at the time the SPF and 
Michigan forecasts were made.  Both survey measures have highly 
statistically significant predictive power for future inflation even 
controlling for the inflation rate's recent past history, but the SPF 
measure has substantially more predictive power.  The `horserace' 
regression results indicate that the Michigan survey measure contains 
no information that is not also included in the SPF measure, while the 
SPF forecast has highly statistically significant predictive power 
that is not contained in the Michigan survey.\footnote{A more 
stringent test would be whether the surveys can predict the {\it 
change} in the inflation rate.  See 
Carroll~\cite{carroll:epidemicinflNBER} for this test, which again 
finds that both surveys have highly significant predictive power but 
the SPF has more power.  A more extensive evaluation of the 
forecasting power of the indexes is provided in the archive of 
programs that generated all of the results in this paper, available on 
the author's website.} Note that this result implies that the Michigan 
forecast is {\it prima facia} irrational (using the usual definition 
in rational expectations models), since the information that 
forecasters possessed that allowed them to make a superior forecast 
was in principle also available to households.  Thus, we can 
unambiguously conclude that the SPF forecast is `more rational' than 
the Michigan forecast, and the difference is large in both statistical 
and economic terms.

% Next idea: Granger causality
A final preliminary check is suggested by the structure of the model, 
in which expectations are assumed to spread from forecasters to 
households.  This suggests that the professional forecasts should 
Granger-cause the household forecasts, but not vice versa.  
Table~\ref{table:granger} shows that there is indeed statistical 
evidence of Granger causality from the professional forecasts to 
household forecasts, but no Granger causality in the opposite 
direction.

\ifthenelse{\boolean{FigsAndTablesInText}}{\input :DataAndPrograms:Data:granger.tex}{}

% Is SPF fully rational?  Who cares?
Of course, a finding that the SPF forecast is better than the Michigan 
forecast does not necessarily imply that the SPF forecast is fully 
rational.  However, Croushore~\cite{croushore:spfisgood} reports 
results for a battery of optimality tests proposed in the {\it 
Handbook of Statistics} by Diebold and 
Lopez~\cite{diebold&lopez:forecastbias}; the SPF fails only one of 
these tests, the DuFour test, which is actually partly a test of the 
symmetry of the forecast errors around zero.  Since nothing in 
rational expectations theory requires errors to be symmetrically 
distributed, this test is arguably of less interest than the other 
tests.  Finally, note that the question of the rationality of the SPF 
forecasts is logically separate from the enterprise here, which is to 
examine whether the Michigan forecasts can be well modeled as updating 
toward the SPF forecasts.  Rationality of the SPF forecasts is 
interesting in and of itself, but is in principle an independent 
question that can be addressed separately (as in 
Croushore~\cite{croushore:spfisgood}).

\subsection{Estimating the Stickiness of Inflation Expectations}

% Does the model work (provide a decent description of data)?
We now turn to the main question, which is whether the Michigan survey 
data can be reasonably well represented by the 
model~\eqref{eq:esteqn}.

\ifthenelse{\boolean{FigsAndTablesInText}}{\input :DataAndPrograms:Data:esteqn.tex}{}

To provide a baseline for comparison, the first line of 
Table~\ref{table:esteqn} presents results for the simplest possible 
model: that the value of the Michigan index $M_{t}[\pi_{t,t+4}]$ is 
equal to a constant, $\alpha_{0}$.  By definition the $\bar{R}^{2}$ is 
equal to zero; the standard error of the estimate is 0.88.  The last 
column of the table is reserved for reporting the results of various 
tests that will be conducted as the analysis progresses.  By way of 
example, the test performed for the benchmark expectations-constant 
model is whether the average value of the expectations index is zero, 
$\alpha_{0} = 0$.  Unsurprisingly, this nonsensical proposition can be 
rejected with an overwhelming degree of statistical confidence, as 
indicated by a $p$-value that says that the probability that the 
proposition is true is zero.

% Estimate unrestricted version
We begin to examine the baseline model's ability to explain the Michigan
data by estimating
\begin{eqnarray}
	M_{t}[\pi_{t,t+4}] & = & 
	\alpha_{1}S_{t}[\pi_{t,t+4}]+\alpha_{2}M_{t}[\pi_{t-1,t+3}]+\epsilon_{t}, \label{eq:firstest}
\end{eqnarray} 
where $S_{t}[\pi_{t,t+4}]$ is the corresponding SPF forecast.  
Comparing this to \eqref{eq:esteqn} provides the testable restriction 
that $\alpha_{2}=1-\alpha_{1}$ or, equivalently,
\begin{equation}
\alpha_{1}+\alpha_{2}=1 \label{eq:restriction}.
\end{equation}

% Restriction easily accepted; when imposed, \lambda =0.27
Results from the estimation of \eqref{eq:firstest} are presented as 
equation 1.  The point estimates of $\alpha_{1}=0.36$ and 
$\alpha_{2}=0.66$ suggest that the restriction \eqref{eq:restriction} 
is very close to holding true, and the last column presents formal 
statistical evidence on the question: It shows that the statistical 
significance with which the proposition that $\alpha_{1}+\alpha_{2}=1$ 
can be rejected is only about 0.18, so that the restriction is easily 
accommodated by the data at the conventional level of significance of 
0.05 or greater.  Estimation results when the restriction is imposed 
in estimation are presented in the next row of the table, which 
provides our first unambiguous estimate of the crucial coefficient: 
$\lambda= 0.27$.  Note that the Durbin-Watson statistic indicates that 
there is no evidence of serial correlation in the residuals (a Q-test 
yields the same result), which is impressive because the individual 
series involved have very high degrees of serial correlation.  This is 
evidence that the two variables are cointegrated, as would be expected 
if one were a distributed lag of the other.

% Estimate is 0.27 cf M&R 0.25
The point estimate $\lambda=0.27$ is remarkably close to the value of 
$0.25$ assumed by Mankiw and Reis~\cite{mankiw&reis:slumps,mankiw&reis:stickye} in their 
simulation experiments; unsurprisingly, the last column for equation 2 
indicates that the proposition $\alpha_{1}=\lambda=0.25$ is easily 
accepted by the data.  This estimate indicates that in each quarter, 
only about one fourth of households have a completely up-to-date 
forecast of the inflation rate over the coming year.  On the other 
hand, it also indicates that only about 32 percent ($=(1-0.25)^{4})$ 
of households have inflation expectations that are more than a year 
out of date.

% Comparison to Roberts
As noted above, Roberts~\cite{roberts:inflexp} estimated a similar 
equation, except that his proposal was that expectations move toward 
the mathematically rational forecast of inflation rather than toward the SPF 
measure.  Since the `rational' forecast is unobservable, he used the 
actual inflation rate and instrumented using a set of predetermined 
instruments, on the usual view that if the instruments are valid the 
estimation should yield an unbiased estimate of the coefficient on the 
true but unobservable rational forecast.  However, this procedure is 
problematic if there was anything that the `rational' forecaster did 
not know about the structure of the economy and had to learn from 
realizations over time; as Roberts acknowledges, it is also 
problematic if the structure of the economy changes over time.  A 
final drawback to this approach is that instrumenting can cause a 
severe loss of efficiency.  Since the theory proposed here is quite 
literally that household expectations move toward the SPF forecast, 
there is no reason to instrument.  In the end, however, Roberts's 
parameter estimates are similar to those obtained here, though 
with considerably larger standard errors.

% If you force the data to fit the model, they say OK
What equation 2 of Table~\ref{table:esteqn} indicates is that if the 
data are forced to choose an $\alpha_{1}=1-\alpha_{2}$ they are happy 
with that restriction, and that a model that imposes the restriction 
has a highly statistically significant ability to fit the data.  
However, we have not allowed the data to speak to the question of 
whether there is a better representation of inflation expectations 
than \eqref{eq:firstest}.

% But if we let a constant in, it wants to be there and positive
The first avenue by which we might wish to let the data reject the 
specification is to allow a constant into the equation.  Equation 3 
presents the results.  The last column indicates that the proposition 
that the constant term is zero can be rejected at a very high level of 
statistical significance; on the other hand, the improvement in fit 
that comes with a constant is rather modest: The standard error 
declines from about 0.43 without the constant to about 0.35 with it.  
Compared with a raw standard error for the dependent variable of about 
0.88, this improvement in fit is not very impressive, even if it is 
statistically significant.

% But a model with a positive constant doesn't make sense
Furthermore, if the model is to be treated as a structural description
of the true process by which inflation expectations are formed, the
presence of a constant term does not make much sense.  It implies, for
example, that if both actual inflation and the rational forecast for
inflation were to go to zero forever, people would continue to expect
a positive inflation rate (of a bit under 2 percent) forever.  It
seems much more likely that under these circumstances people would
eventually learn to expect an inflation rate of zero.  This point can
be generalized to show that if the actual inflation rate and the
rational forecast were fixed forever at {\it any} constant value,
people's expectations would never converge to the true, constant
inflation rate, but instead would be perpetually biased (unless the
true value happened to be exactly equal to the unique stable point of
the estimated equation).

% The more elaborate epidemiological model can explain a constant
A more palatable explanation for the presence of a constant term is
that the baseline model is not a perfectly accurate
description of the process by which information is transmitted in the
economy; in this case estimation of the misspecified model could
result in a spuriously significant coefficient term.  For example,
Carroll~\cite{carroll:epidemicinflSFI} demonstrates that the presence
of a significant constant term could reflect the presence of some
social transmission of inflation expectations via conversations with
neighbors (epidemiologically, the disease is locally communicable), in
addition to the news-media channel examined here.

% Allowing adaptive rather than rational expectations doesn't work
Another plausible modification to the model is to allow for the
possibility that some people update their expectations to the most
recent {\it past} inflation rate rather than to the SPF forecast of
the future inflation rate.  Since most news coverage of inflation is
prompted by the release of past inflation statistics (and since the
new past number is often in the headline of the news article) one
might argue that it might seem {\it more} likely for people to update
their expectations to the past inflation rate than to a forecast of
the future rate.  This coresponds to what is usually called a model of
`adaptive expectations.'  As noted above, however, if people believe
that the true inflation process is as described in
\eqref{eq:Fpi}-\eqref{eq:Fpi2}, this adaptive expectations benchmark
is also identical to the limited-information rational expectations
forecast (again, remember that we are assuming that households believe
professional forecasters know much more about the inflation process
than is contained in its past history, so updating households could
still believe that the SPF forecast is better than the adaptively
rational forecast would be).

We can examine these possibilities by estimating an equation of the
form
\begin{eqnarray}
	M_{t}[\pi_{t,t+4}] & = & \alpha_{1} S_{t}[\pi_{t,t+4}]+\alpha_{2} M_{t-1}[\pi_{t-1,t+3}]+\alpha_{3} P_{t}[\pi_{t-5,t-1}],
\end{eqnarray}
where $P_{t}[\pi_{t-5,t-1}]$ represents the most recently published 
annual inflation rate as of time $t$.

% Allow horserace with adaptive expectations
Results from estimating this equation are presented in the next row of
Table~\ref{table:esteqn}.  The past inflation rate is indeed highly
statistically significant - but with a {\it negative} coefficient! 
The negative coefficient makes no sense, as it implies that a higher
past inflation rate convinces people that the fundamental inflation
rate is lower.  The final row of the table, however, shows that when a
constant is included in this regression, the past inflation rate is no
longer statistically significant, while the forecast of the future
inflation rate remains highly statistically significant.  This seems
to indicate that the significance of the past inflation rate is
spurious, in the sense that the past inflation rate is just proxying
for the missing constant term, which we have already acknowledged to
be statistically significant.  The last row in the table shows,
surprisingly, that even when the SPF forecast is entirely absent, the
lagged inflation rate has no explanatory power for the Michigan survey
after controlling for the survey's own lagged value; furthermore, the
Durbin-Watson statistic suggests a substantial amount of negative serial
correlation in the residuals of this equation, in contrast with the
baseline model.



% Model does pretty well
In sum, it seems fair to say that the simple `sticky expectations'
equation~\eqref{eq:esteqn} does a remarkably good job of capturing
much of the predictable behavior of the Michigan inflation
expectations index.\footnote{One further robustness test is presented
in Carroll~\cite{carroll:epidemicinflNBER}: Estimation of the model on
monthly rather than quarterly data.  This gets around the timing
problems caused by the fact that the Michigan index for a quarter
reflects interviews continuously throughout the quarter while the SPF
reflects forecasters' views at a point in time roughly halfway through
the quarter.  Estimates of the monthly $\lambda$ are roughly a third
the size of estimates of the quarterly rate, as would be expected if
the quarterly estimates were unbiased.}

\subsection{Inflation News Coverage and Inflation Expectations}

% When there are more news stories, people should be better informed
If we take literally the assumption that people derive their inflation
expectations from news stories, we should expect that when there are
more news stories people should be better informed. 
Figure~\ref{fig:inflfigs} plots an index of the intensity of news
coverage of inflation in the New York Times and the Washington Post
against the actual inflation rate;\footnote{The index was constructed
as follows.  For each newspaper $i \in \{\mbox{New York
Times},\mbox{Washington Post}\}$, for each year $t$ since 1980 (when
the Nexis index of both newspapers begins), a search was performed for
stories that began on the front page of the newspaper and contained
words beginning with the root `inflation' (so that, for example,
`inflationary' or `inflation-fighting' would be picked up).  For each
newspaper, the number of stories was converted to an index ranging
between zero and 1 by dividing the number of stories in a given year
by the maximumn number of inflation stories in any year.  Thus, the
fact that the overall index falls to about 0.25 in the last part of
the sample indicates that there were about a quarter as many
front-page stories about inflation in this time period as there were
at the maximum.} unsurprisingly, the intensity of news coverage of
inflation was highest in the early 1980s when the actual inflation
rate was very high, and inflation coverage has generally declined
since then.  Note, however, that the actual inflation rate fell
farther and faster than the news index; evidently, inflation remained
an important story during the period when it was dropping rapidly.

\ifthenelse{\boolean{FigsAndTablesInText}}{\input :DataAndPrograms:Data:inflfigs.tex}{}

%\begin{figure} \centerline{\includegraphics[scale=0.7]{:DataAndPrograms:Data:inflmicvsspf.eps}}%\epsfbox{:DataAndPrograms:Data:inflindex.eps}} \caption{Michigan Versus SPF Forecasts of Inflation} \label{fig:inflmicvsspf}\end{figure}

% Plot actual vs forecasts of infl
The bottom panel of Figure~\ref{fig:inflfigs} plots the SPF and 
Michigan forecasts since the third quarter of 1981 when the SPF first 
began to include CPI inflation.  One striking feature of the figure is 
that during the high-news-coverage period of the early 1980s, the size 
of the gap between the SPF forecast and the Michigan forecast is 
distinctly smaller than the gap in the later period when there was 
less news coverage of inflation.

% Formal test of whether GAP is related to NEWS
A formal statistical test of whether greater news coverage is 
associated with `more rational' household forecasts (in the sense of 
forecasts that are closer to the SPF forecast) can be constructed as 
follows.  Defining the square of the gap between the Michigan and SPF 
forecasts as GAPSQ$_{t} = (M_{t}-S_{t})^{2}$, and defining the 
inflation index as NEWS$_{t}$, we can estimate the simple OLS 
regression equation
\begin{eqnarray}
\mbox{GAPSQ}_{t} & = & \alpha_{0} + \alpha_{1} \mbox{NEWS}_{t}.
\end{eqnarray}

% Gap is indeed lower when news is higher
Table~\ref{table:erronnews} presents the results.  Estimated over the 
entire sample from 1981q3 to 2000q2 the regression finds a negative 
relationship that is statistically significant at the 5 percent level 
after correcting for serial correlation.  The second row shows that 
that if the first year of the SPF CPI forecasts is excluded the 
negative relationship is much stronger and statistically significant 
at better than the 1 percent level; however, aside from the 
possibility that the first few SPF CPI forecasts were problematic in 
some way, there seems to be little reason to exclude the first year of 
SPF data.

\ifthenelse{\boolean{FigsAndTablesInText}}{\input :DataAndPrograms:Data:erronnews.tex}{}

% Test direct implication: higher infection rate when more stories
The finding that household inflation forecasts are better when there
is more news coverage is an indirect implication of the model under
the assumption that absorption of the SPF forecast is more likely when
there is more inflation coverage.  The proposition that the absorption
rate is higher when there are more news stories can also be tested
directly.  Table~\ref{table:newslam} presents estimation results
comparing the absorption rate estimated during periods when there is
more news coverage than average (NEWS$_{t}>$mean(NEWS)) and less
coverage than average (NEWS$_{t}<$mean(NEWS)).  The estimate of
$\lambda$ is almost 0.7 during periods of intensive news coverage, but
only about 0.2 during periods of less intense coverage; an F-test
indicates that this difference in coefficients is statistically
significant at the 5 percent level (and nearly at the 1 percent
level).

\ifthenelse{\boolean{FigsAndTablesInText}}{\input :DataAndPrograms:Data:newslam.tex}{}

% Existing lit: ADP
There are several strands of the existing literature that deserve
comment at this point.  In two important recent papers, Akerlof,
Dickens, and Perry~\cite{adp:one,adp:two} have proposed a model in
which workers do not bother to inform themselves about the inflation
rate unless inflation gets high enough that ignorance would become
costly.  Since periods of high news coverage have coincided with
periods of high inflation, this model is obviously consistent with the
finding that mean inflation expectations are more rational during
periods of high coverage.  Indeed, in a way the ADP models are deeper
than the one proposed here, because they provide an explanation for
the intensity of news coverage which is taken as exogenous here: The
news media write more stories on inflation in periods when workers are
more interested in the topic.

% Existing lit: Roberts
These results can also be viewed as somewhat similar to some findings 
by Roberts~\cite{roberts:inflexp}, who estimates a model 
like~\eqref{eq:esteqn}, performs a sample split, and finds the speed 
of adjustment parameter much larger in the post-1976 period than in 
the pre-76 era.  He interprets this as bad news for the model.  
However, the pre-76 era was one of much more stable inflation (until 
the last years) than the post-76 era, so the finding of a higher 
coefficient in the later years is very much in the spirit of the tests 
performed above, and is therefore consistent with the 
interpretation of the model proposed here.

% BMR and DKW


\section{Unemployment Expectations}

% To be generally useful, needs to work for things other than inflation
If the model of expectations proposed here is to be generally useful 
to macroeconomists, it will need to apply to other variables in 
addition to inflation.  Another potential candidate is unemployment 
expectations; in previous work (Carroll~\cite{carroll:brookings}, 
Carroll and Dunn~\cite{carroll&dunn:macroannual}) I have found 
unemployment expectations to be a powerful predictor of household 
spending decisions, and since household spending accounts for two 
thirds of GDP, understanding the dynamics of unemployment expectations 
(and any deviations from rationality) should have considerable direct 
interest.

% Michigan survey question doesn't ask the level of unemployment
Unfortunately, however, the Michigan survey's question on unemployment 
does not ask households to name a specific figure for the future 
unemployment rate; instead, households are asked whether they expect 
the unemployment rate to rise, stay the same, or fall over the next 
year.  Traditionally, the answers to these questions are converted 
into an index by subtracting the ``fall'' from the ``rise'' 
proportion.  This diffusion index can then be converted into a 
forecast of the change in the unemployment rate by using the predicted 
value from a regression of the actual change in unemployment on the 
predicted change.

% Explain construction of unemployment forecast
That is, the regression
\begin{eqnarray}
\bar{U}_{t,t+4}-\bar{U}_{t-4,t} & = & \gamma_{0} + \gamma_{1}M_{t}^{U} \label{eq:duonmichu}
\end{eqnarray}
is estimated, where $\bar{U}_{t,t+4}$ is the average unemployment rate 
over the next year and $\bar{U}_{t-4,t}$ is the unemployment rate over 
the year to the present, and $M_{t}^{U}$ is the Michigan index of 
unemployment expectations.  With the estimated 
$\{\hat{\gamma}_{0},\hat{\gamma}_{1}\}$ in hand a forecast of next 
year's unemployment rate can be constructed from
\begin{eqnarray}
\hat{\bar{U}}_{t,t+4} & = & \hat{\gamma}_{0} + \hat{\gamma}_{1}M_{t}^{U}+\bar{U}_{t-4,t}.
\end{eqnarray}

% MU has lots of power for U, but not when SPF is controlled for
When~\eqref{eq:duonmichu} is estimated, the coefficient on $M_{t}^{U}$ 
is has a t-statistic of over 8, even after correcting for serial 
correlation.  However, in a horserace regression of the actual change 
in unemployment on the Michigan diffusion index and the SPF forecast 
of the change in unemployment, the Michigan forecast has no predictive 
power.  Thus, as with inflation, it appears that on average people 
have considerable information about how the unemployment rate is 
likely to change, but forecasters know a lot more than households do.

% Results for U analogy to esteqn 
Table~\ref{table:esteqnunemp} presents a set of regression results for 
the household unemployment forecast that is essentially identical to 
the tests performed in Table~\ref{table:esteqn} for inflation 
expectations.

\ifthenelse{\boolean{FigsAndTablesInText}}{\input :DataAndPrograms:Data:esteqn_unemp.tex}{}

% Works great
The point estimate of the speed of adjustment parameter in row 3 is
$\alpha_{1}=0.31$; the test reported in the last column of that row
indicates that this is statistically indistinguishable from the
estimate of $\lambda=0.25$ obtained for inflation expectations.  In
most respects, in fact, the model performs even better in explaining
unemployment expectations than in explaining inflation expectations.  For
example, row 3 indicates that the equation does not particularly want
a constant term in it, while row 4 finds that the lagged level of the
unemployment rate has no predictive power for current expectations
even when a constant is excluded.

% But interpretation is rather dicey \ldots
Nonetheless, this evidence should be considered with some caution. 
The process of constructing the forecast for the average future level
of the unemployment rate, while apparently reasonable, may be
econometrically and conceptually problematic.  In particular, this
method assumes that the {\it amount} by which unemployment is expected
to change on average is related to the {\it proportion} of people who
expect unemployment to rise or fall; in fact, there is no necessary
linear relationship between these two quantities.  Other econometric
difficulties may come from the use of constructed variables on both
the left and right hand sides of the equation.  I view this model of
unemployment expectations merely as secondary supporting evidence for
the expectations modeling strategy pursued here, and therefore am not
inclined to pursue these conceptual and econometric problems further,
though they might be worth pursuing in later work.

\section{Discussion and Relationship to Existing Literature}

% Why HH exps matter: wage-push inflation

A potential criticism of this paper might be that expectations of
households are unimportant for macroeconomic outcomes; instead,
perhaps what matters are expectations of experts, which may be
rational in the traditional sense.  This is not a plausible criticism
with respect to unemployment expectations, given the powerful
influence that households' unemployment expectations have on household
spending (Carroll and Dunn~\cite{carroll&dunn:macroannual});
households' consumption decisions surely depend on their own views
rather than the views of others.  In the case of inflation
expectations, whether households' expectations are important
presumably depends on whether sluggish household expectations are
partly or largely responsible for the costly nature of disinflations. 
This seems a good guess, since credible preannounced disinflations
should be costless (or nearly so) in an economy in which all agents
have rational expectations (Ball~\cite{ball:sacrifice}).  It seems
likely that a good part of the influence of household expectations on
inflation comes through a labor market channel.  Standard models of
unemployment, either in the search literature or the efficiency wage
literature, almost always rely upon an assumption that households'
labor supply decisions are made by judging the appeal of available or
expected real wage offers.  It is an empirical fact that wage
contracts are usually written in nominal terms, so these models of
cyclical and structural unemployment entail an implicit assumption
that households can translate nominal wage offers into real ones,
which requires them to have expectations about inflation.  Since wages
are around two thirds of business costs, if households' inflation
expectations affect nominal wage outcomes they must affect firms'
pricing decisions through the usual `wage-push' channel.

% Relationship to credibility literature

The potential importance of household expectations provides a new
perspective on the ongoing debate about the importance of
`credibility' in monetary policy.  Credibility has usually been
thought of in terms of the beliefs of policy experts and private
forecasters; for example, an extensive recent treatment in Bernanke
et.  al.~\cite{blmp:infltarget} judges credibility of inflation
targeting regimes by examining how quickly professional forecasts
converge to the stated target range.  Indeed, some central banks (such
as the Bank of England and the Bank of Israel) have begun to
officially look at surveys of forecasters as well as yields on
inflation indexed bonds as measures of such ``expert'' opinion. 
However, empirical tests of whether credibility matters for monetary
policy have produced mixed results; see, e.g.,
Posen~\cite{posen:cred2,posen:cred} or Debelle and
Fischer~\cite{debelle&fischer:cred}, and for an insider's perspective
and an excellent summary of the literature see
Blinder~\cite{blinder:centbank}.  The results above, however, indicate
that there are substantial gaps between beliefs of forecasters and of
households, so credibility among experts may not be sufficient to
achieve a desired inflationary outcome; the views of the experts need
to be communicated effectively to the population to become effective.

% Add price stickiness too

In many cases the model is likely to yield similar, though not
identical, conclusions to those obtained from models with price
stickiness.  Consider, for example, the finding of Ball, Mankiw, and
Romer~\cite{bmr:phillips} that the Phillips curve is steeper when
inflation is higher, which they attribute to a reduction in price
stickiness at high inflation rates (a further justification of such
effects can be found in the recent paper by Dotsey, King, and
Wolman~\cite{dkw:Ss}).  If the newsworthiness of inflation forecasts
depends on the level of inflation and $\lambda$ reflects the intensity
of news coverage, then very similar implications could be derived in
this model.  However, the two interpretations could be distinguished
if newsworthiness is related to the {\it change} in inflation (as
figure~\ref{fig:inflfigs} suggests).

A related question is why it appears to be easier to end high
inflations than moderate ones
(Sargent~\cite{sargent:biginfl,sargent:modinfl}): Perhaps the attention
of the population tends to be intensely focused on government
inflation-fighting policies during hyperinflations (so $\lambda = 1$
and the model collapses to rational expectations), while attention may
be focused on other matters during attempts to end moderate
inflations.  Ball~\cite{ball:sacrifice} shows that quicker
disinflations seem to entail smaller sacrifices of output; again, this
could reflect the fact that the policies needed to achieve a quick
disinflation are more dramatic, and therefore more newsworthy, than in
a gradualist approach.  One way of testing these ideas for more
recent episodes would be to construct indices of news coverage of
inflation like those presented above for other countries during
disinflationary episodes.

%
Of course, the real world presumably combines some degree of price
stickiness and a degree of expectational stickiness.  The results in
Ball~\cite{ball:credibility} showing strong interactions between
credibility and price stickiness suggest that a model that combines
sticky expectations and sticky prices might generate results different
from the results obtainable with either feature alone; this would also
be an interesting topic for future research.

\section{Conclusions}

Given the consensus among economists that macroeconomic outcomes
depend critically on agents' expectations, it is surprising that there
has been very little effort to test positive models of expectations
using the large body of empirical expectations data available from the
Michigan Survey and the Conference Board.

This paper shows that a very simple model in which the typical
household's expectations are updated probabilistically toward the
views of professional forecasters does a good job of capturing much of
the variation in the Michigan Survey's measures of inflation and
unemployment expectations.  In addition to fitting these data, the
model can be interpreted as providing microfoundations for the
aggregate expectations equation postulated recently by Mankiw and
Reis~\cite{mankiw&reis:slumps,mankiw&reis:stickye} and earlier by
Roberts~\cite{roberts:inflexp}.  As those papers show, macroeconomic
dynamics are more plausible in a variety of dimensions, including the
tradeoff between inflation and unemployment, the reaction of the
economy to monetary shocks, and the relationship between productivity
growth and unemployment, when expectations deviate in this way from
the rational expectations benchmark.

There are many directions in which research could fruitfully proceed
from here.  First, the Michigan and Conference Board surveys contain
many other expectational variables that could be studied to see
whether the specification proposed here is widely applicable.  Second,
the model could be tested at the micro level using the raw
household-level data from the surveys.  (One approach would be to
extend the model of Branch~\cite{branch:hetero}, in which individuals
choose among various alternative predictors, to include the SPF
forecast among the competing predictors).  Third, the implications of
more sophisticated models of the spread of expectations in the
population can be examined (see Carroll~\cite{carroll:epidemicinflSFI}
for a start).  And much more work remains to be done to investigate
the empirical and theoretical properties of macroeconomic models in
which expectations are formed in this way.

Finally, it is clear that in order for this framework to be a complete
and general purpose tool, it will be necessary to develop a theory
that explains the variations in the absorption parameter $\lambda$
over time.  For present purposes it was enough to show that $\lambda$
is related to the intensity of news coverage, but that only pushes the
problem one step back, to the need for a model of the extent of news
coverage.  Possibly the approach offered in a recent paper by
Sims~\cite{sims:inattention} could help; Sims examines models of
`rational inattention' which imply that an agent with limited
information processing capacity should optimally ignore most
macroeconomic data.  The difficulty with applying the Sims framework
to consumers directly is that solving the problem of deciding what to
ignore is even harder than solving the full-information rational
expectations model.  However, if the news media were viewed as the
agents who solve the information compression problem (since the
information stream they can convey is obviously limited), the Sims
framework might provide a useful formal model of how the news media go
about deciding how much coverage to give to economic matters.

\medskip\medskip\medskip{\sc The Johns Hopkins University}

\input MacPocket:Software:latex:texhtml


\pagebreak\vfill\eject


\ifthenelse{\boolean{QJETypesetVersion}}{
   \newpage\begingroup
   \setlength{\parindent}{0pt}
   \setlength{\parskip}{2ex}
%   \def{\enotesize}{\normalsize}
   \theendnotes
   \endgroup
}

\pagebreak\vfill\eject


\bibliographystyle{MacPocket:Software:latex:qje}
\bibliography{MacPocket:Software:latex:economics}


\ifthenelse{\boolean{FigsAndTablesInText}}{}{
\pagebreak
%\input :DataAndPrograms:Data:ADFpi.tex
\input :DataAndPrograms:Data:piforc.tex
\input :DataAndPrograms:Data:granger.tex
\input :DataAndPrograms:Data:esteqn.tex
\input :DataAndPrograms:Data:erronnews.tex
\input :DataAndPrograms:Data:newslam.tex
\input :DataAndPrograms:Data:esteqn_unemp.tex
%\input :DataAndPrograms:Data:estonsim.tex
%\input :DataAndPrograms:Data:estran.tex
%\input :DataAndPrograms:Data:estnbr.tex

\pagebreak
\input :DataAndPrograms:Data:inflfigs.tex
%\input :DataAndPrograms:Data:inflstd.tex
%\input :DataAndPrograms:Data:inflstdlogerr.tex
}


\end{document}


\pagebreak\vfill\eject


The ancient Greek philosopher Xenophanes once observed that if oxen 
had artistic talent, the gods they would paint would look like oxen 
instead of like Apollo or Aphrodite.  

Xenophanes might not have been surprised to learn that the agents with 
whom economists populate their models are essentially godlike versions 
of economists: infinitely knowledgable and infinitely rational and 
deeply interested in the structure and workings of the economy.  
However, in a twist that surely would have surprised even Xenophanes, 
we assume that these godlike creatures are a good representation of 
the average person or business manager.

\begin{eqnarray}
	M_{t}[\pi_{t+1}] & = & \lambda N_{t}[\pi_{t+1}] + (1-\lambda)\lambda E_{t-1}[\pi_{t+1}]+(1-\lambda)^{2} \lambda E_{t-2}[\pi_{t+1}] + \ldots .
\end{eqnarray}
\begin{eqnarray*}
	\pi_{t-4,t} & = & \pi_{t-3}+\pi_{t-2}+\pi_{t-1}+\pi_{t}  \\
	 & = & F_{t-3}[\pi_{t-3}]+\epsilon_{t-3}+F_{t-2}[\pi_{t-2}]+\epsilon_{t-2}  
 +F_{t-1}[\pi_{t-1}]+\epsilon_{t-1}+F_{t}[\pi_{t}]+\epsilon_{t}
\\   & = & 4 F_{t-3}[\pi_{t-3}]	+ 3 \eta_{t-2}+2\eta_{t-1}+\eta_{t}
+ \epsilon_{t-3}+\epsilon_{t-2}+\epsilon_{t-1}+\epsilon_{t}
\end{eqnarray*}
and analogously 
\begin{eqnarray*}
	\pi_{t,t+4} & = & 4 F_{t+1}[\pi_{t+1}]	+ 3 \eta_{t+2}+2\eta_{t+3}+\eta_{t}
+ \epsilon_{t+1}+\epsilon_{t+2}+\epsilon_{t+3}+\epsilon_{t+4}
\\ & = & 4 F_{t-3}[\pi_{t-3}]+4(\eta_{t-2}+\eta_{t-1}+\eta_{t}+\eta_{t+1})+ 3 \eta_{t+2}+2\eta_{t+3}+\eta_{t}
\\ &   & + \epsilon_{t+1}+\epsilon_{t+2}+\epsilon_{t+3}+\epsilon_{t+4}
\end{eqnarray*}

Consider how 
the inflation rate



If inflation is indeed a unit root process with a white noise 
transitory component, then the inflation rate over the next 
year will be
\begin{eqnarray*}
	\pi_{t,t+4} & = & F_{t}[\pi_{t+1}]+\epsilon_{t+1}+F_{t+2}[\pi_{t+2}]+\epsilon_{t+2}
+ F_{t+3}[\pi_{t+3}]+\epsilon_{t+3}+F_{t+4}[\pi_{t+4}]+\epsilon_{t+4}	
\\	 & = & 4 F_{t}[\pi_{t}]+4\eta_{t+1}+3 \eta_{t+2}+2\eta_{t+3}+\eta_{t+1}+\epsilon_{t+1}+\epsilon_{t+2}+\epsilon_{t+3}+\epsilon_{t+4}
%\\	 & = & F_{t}[\pi_{t-1,t}]+\chi_{t,t+4}
\end{eqnarray*}
where 
$E_{t}[\eta_{t+1}]=E_{t}[\eta_{t+2}]=E_{t}[\epsilon_{t+1}]=E_{t}[\epsilon_{t+2}]=\ldots=0$ 
implies that $\chi_{t,t+4}$ is a white noise error term.  That is, the 
only element of predictability in future inflation rates comes from 
the fact that the future fundamental rate is equal to the current 
fundamental rate plus some white noise innovations.




Now consider a regression
of the inflation rate over the next year on the inflation rate for the
current quarter:
\begin{eqnarray}
	\pi_{t,t+4} & = & \alpha_{0}+\alpha_{1} \pi_{t-1,t}
\\  F_{t}[\pi_{t-1,t}]+\chi_{t,t+4}            	& = & \alpha_{0}+\alpha_{1} (F_{t}[\pi_{t-1,t}]+\epsilon_{t})
\\  F_{t}[\pi_{t-1,t}]            	& = & \alpha_{0}+\alpha_{1} (F_{t}[\pi_{t-1,t}]+\epsilon_{t})-\chi_{t,t+4}
\end{eqnarray}

If the variance of the transitory component of inflation were zero, 
the regression should 


 then the only 
predictable element of the difference between inflation over the next 
year and inflation in the current quarter would come from the 
$-4\epsilon_{t}$ term.  

Table~\ref{table:survforecast} presents evidence on these questions.  
The first two rows show that both the Michigan inflation expectations 
index and the SPF forecast have highly statistically significant 
predictive power for the core inflation rate, even when the lagged 
core inflation rate is included as a regressor; indeed, in neither 
case is the lagged inflation rate statistically significant, which 
indicates that both survey measures already incorporate any 
information contained in the lagged inflation rate.\footnote{The 
Durbin-Watson and Q statistics indicate that there is a substantial 
amount of serial correlation in the residuals from these equations, so 
the standard errors are adjusted using the Newey-West correction with 
8 lags and a damping factor of 1.}$^{,}$\footnote{It might seem that 
this evidence that the Michigan index can forecast the change in the 
inflation rate contradicts the assumption that people believe 
inflation follows a random walk.  In fact, however, there is no 
contradiction.  In our model, people believe that future 
fundamental inflation rates follow a random walk beginning in period 
$t+1$.  But their views of the inflation rate for period $t+1$ are not 
constrained to be a random innovation with respect to the actual 
inflation rate in period $t$, for two reasons.  First, we assumed that 
people in period $t$ believe that $F_{t}[\pi_{t+1}]$ is equal to 
the rational forecast $E_{t}[\pi_{t+1}]$, and there is no restriction 
on how the rational forecast may differ from the recent level of 
inflation; in particular, the rational forecast may be influenced, for 
example, by the level of the unemployment rate.  Second, people in 
this model know that any transitory inflation shocks in period $t$ 
will disappear for future periods.} The third row of the table shows 
that when both survey measures are included in the forecasting 
equation, the Michigan survey measure has no information about the 
future inflation rate that is not contained in the SPF forecast, while 
the SPF forecast has considerable and highly statistically significant 
predictive power not contained in the Michigan forecast.  The SPF's 
victory is made more impressive by the fact that the SPF forecast is 
constructed in the middle of the second month of the quarter, and 
therefore by construction cannot incorporate any information released 
during the latter half of the quarter, while the Michigan survey 
measure for the corresponding quarter includes results from interviews 
up to the last day of the quarter.  The fact that the SPF contains 
considerably more forecasting power despite its substantial timing 
disadvantage bolsters the proposition that the SPF forecast is 
considerably better than the Michigan forecast.

An alternative way to examine the rationality and efficiency of the 
two forecasts is to ask about the ability of the two surveys to 
forecast the {\it change} in the inflation rate.  In a statistical 
sense this can be viewed as a tougher challenge than forecasting the 
level of the inflation rate; after all, if the inflation rate really 
has a unit root (which we could not reject above), then the level of 
the inflation rate would be highly forecastable but changes in the 
inflation rate (beyond the current quarter) would be totally 
unforecastable.  The next three panels of the table therefore regress 
the change in the annual inflation rate between the present and one 
year in the future on the SPF and Michigan survey's forecasts of that 
change.  That is, 

\begin{eqnarray*}
	\pi_{t-4,t} & = & \pi_{t-3}+\pi_{t-2}+\pi_{t-1}+\pi_{t}  \\
	 & = & F_{t-3}[\pi_{t-3}]+F_{t-2}[\pi_{t-2}]+F_{t-1}[\pi_{t-1}]+F_{t}[\pi_{t}]  \\
	 &  & + \epsilon_{t-3}+\epsilon_{t-2}+\epsilon_{t-1}+\epsilon_{t}
\end{eqnarray*}

\begin{eqnarray*}
	F_{t}[\pi_{t,t+4}] & = & 4 F_{t}[\pi_{t+1}]
\end{eqnarray*}

Comparing these two expressions, it is evident that there are 
several

Again, both surveys contain highly statistically significant 
information, but again the SPF forecast is substantially better than 
the Michigan forecast.\footnote{In principle, such predictability 
could reflect either time aggregation problems or the modest 
predictability that comes from knowing that the current transitory 
components of inflation will disappear.  The }


Now we can integrate the fact that experts have some ability to 
predict the future inflation rate with the person's view that the 
inflation process has a unit root by supposing that when people 
update their views about the fundamental inflation rate they update to 
what they perceive to be the fully-rational forecast of the 
fundamental rate.



  
That is, if we denote the fully-rational forecast of next quarter's 
inflation rate as $E_{t}[\pi_{t+1}]$, people who are updating their 
forecasts in quarter $t$ set
\begin{eqnarray}
	F_{t}[\pi_{t,t+1}] & = & E_{t}[\pi_{t,t+4}].
\end{eqnarray}



This assumption can be justified by noting that newspaper articles on 
inflation almost invariably incorporate interviews with economists who 
are experts on forecasting the inflation rate.  While the news event 
that triggers the article is likely to have been the release of the 
latest inflation statistics, the job of a good journalist is to 
provide perspective and to extract the meaning of those statistics, 
which is done largely through interviews with experts.  Furthermore, 
an examination ofa sample of these articles indicates that the most 
common form of forecast is for the inflation rate over the next year; 
it is rare for expert forecasts of any particular future quarter to 
appear.  It is not implausible, therefore, to suppose that at least 
part of what readers take away from such stories is a sense of what to 
expect for the average inflation rate over the next year, which (under 
people's views of the inflation process) should be the same as the 
inflation forecast for the next quarter.

people's belief in a unit root in the inflation process implies that
\begin{eqnarray}
	F_{t}[\pi_{t+2}] & = & F_{t}[\pi_{t+1}] 
\\  F_{t}[\pi_{t+3}] & = & F_{t}[\pi_{t+2}] = F_{t}[\pi_{t+1}]	
\end{eqnarray}
and so on.

% Consider putting in the fact that the SPF forecast of the quarterly
% inflation rate has a mean absolute change of 0.3 from tp1 to tp4,
% and 0.2 from tp2 to tp4.  This shows that a constant F is not
% implausible


At this point it will be useful to introduce some simplifying 
notation.  Denote the inflation rate (expressed at an annual rate) 
between period $t$ and period $t+n$ as $\pi_{t,t+n}$.  Thus what we 
have previously written as $\pi_{t+1}$ becomes $4 \pi_{t,t+1}$, where 
the 4 is required to convert the quarterly rate to an annual rate and 
the subscripts indicate that inflation is measured as the change in 
the log price level between period $t$ and $t+1$.  Using this 
notation, \eqref{eq:with4} can be rewritten:
\begin{eqnarray}
	M_{t}[\pi_{t,t+4}] & = & \lambda E_{t}[\pi_{t,t+1}]+(1-\lambda) M_{t-1}[\pi_{t-1,t+3}] . \label{eq:esteqn}
\end{eqnarray}


because the quarterly data we have used 
for the Michigan inflation forecast and the SPF forecast do not map 
very well into our model.


This would suggest 
that if month $t$ is an SPF survey month news reports from that month


If we suppose that news reports on inflation continue to reflect fresh 
contacts between reporters and the participants in the SPF between SPF 
reporting intervals, then the evolution of news media reports will 
depend on how often the typical forecaster updates his forecast.  The 
simplest case would be if all forecasters' inflation forecasts were 
adjusted only once per quarter, just prior to the collection of the 
SPF data.  In this case we could expect newspaper reports to reflect 
the latest SPF forecast all the way up to the point where the new SPF 
forecast is collected.  

I have been unable to find systematic information about how often most 
forecasters' forecasts are updated.  The one forecaster whose methods 
are a matter of public record is the Federal Reserve, whose entire 
macroeconomic forecast is updated roughly once every six weeks (to 
coincide with the monetary policy meetings of the Federal Open Market 
Committee).  If other forecasters were on a similar schedule we would 
expect a given SPF forecast to provide an accurate picture of actual 
forecasts only for about 1-1/2 quarters.


To discuss this question clearly we need to introduce a notational 
convention about the monthly value to be assumed for the SPF forecast 
in months for which no SPF forecast is collected.  Define an operator 
$S_{t}[\pi_{t,t+12}]$ which returns the value of the SPF forecast

If we denote the SPF forecast collected in month $t$ as 
$S_{t}[\pi_{t,t+12}]$.

then this assumption would imply that the 
newspaper reports would be captured by
\begin{eqnarray}
    N_{t+1}[\pi_{t+1,t+13}] & = & S_{t}[\pi_{t,t+12}]
\\  N_{t+2}[\pi_{t+2,t+14}] & = & S_{t}[\pi_{t,t+12}]
\\  N_{t+3}[\pi_{t+3,t+15}] & = & S_{t}[\pi_{t,t+12}]
\end{eqnarray}
where the timing is dictated by the fact that the SPF data are not 
published until near end of the month so that it only becomes possible 
for them to be published in news reports beginning in the next month.  
The second and third equations indicate same SPF forecast continues to 
be published until a new forecast is available.


To discuss this question clearly we need to introduce a notational 
convention about the monthly value to be assumed for the SPF forecast 
in months for which no SPF forecast is collected.  Define an operator 
$S_{t}[\pi_{t,t+12}]$ which returns the value of the SPF forecast

If we denote the SPF forecast collected in month $t$ as 
$S_{t}[\pi_{t,t+12}]$.

then this assumption would imply that the 
newspaper reports would be captured by
\begin{eqnarray}
    N_{t+1}[\pi_{t+1,t+13}] & = & S_{t}[\pi_{t,t+12}]
\\  N_{t+2}[\pi_{t+2,t+14}] & = & S_{t}[\pi_{t,t+12}]
\\  N_{t+3}[\pi_{t+3,t+15}] & = & S_{t}[\pi_{t,t+12}]
\end{eqnarray}
where the timing is dictated by the fact that the SPF data are not 
published until near end of the month so that it only becomes possible 
for them to be published in news reports beginning in the next month.  
The second and third equations indicate same SPF forecast continues to 
be published until a new forecast is available.





The simplest assumption we could make about the news media's inflation 
reporting would be that journalists simply look up the latest 
published forecast from the SPF when they compose their news reports.  
This assumption can be criticized on several grounds.  The first is 
that real journalists like to quote individual forecasters rather than 
reprint results of a publicly available survey; it seems plausible, 
therefore, that the news reports about inflation that are printed 
between the middle and the end of a month in which the survey is taken 
will on average reflect the forecasts made by the survey participants 
in that month, because the journalists are probably contacting roughly 
the same people as were contacted by the SPF. Since most inflation 
articles are published shortly after release of the month's inflation 
data and the articles often reflect interviews asking forecasters how 
the latest data affect their inflation outlook, it seems likely that 
most of the inflation articles published in a survey month will 
actually incorporate the information that is subsequently published in 
the SPF. This would imply that if month $t$ is a survey month,
\begin{eqnarray}
    N_{t}[\pi_{t,t+12}] & = & S_{t}[\pi_{t,t+12}]
\\  N_{t+1}[\pi_{t+1,t+13}] & = & S_{t}[\pi_{t,t+12}]
\\  N_{t+2}[\pi_{t+2,t+14}] & = & S_{t}[\pi_{t,t+12}]
\end{eqnarray}

The assumption that news reports keep repeating the last SPF forecast 
until a new one is collected is also questionable.  If forecasters 
update their forecasts more frequently than the SPF data are 
collected, newspaper reports on inflation are likely to reflect this 
new information in the period before the next SPF forecast is 
conducted.  However, in the absence of survey data in the intervening 
months, little can be done about this problem except to note that it 
may introduce some specification error that makes fit of the equation 
deteriorate.

Furthermore, while there 
is typically only a single rational expectations solution to a macro 
model, dropping any element of the rational expectations framework 
often leads to a model with many possible equilibria which are 
difficult to choose among.

Furthermore, it seems 
likely that a rigorous model of calculation-cost-minimization would 
lead to a periodic (`time dependent') solution rather than 
probabilistic updating.

If month $t$ happened to be a month in which a SPF survey was 
conducted, it seems likely that 


No similar equation would apply for month $t+2$ or $t+3$ because the
SPF forecast is collected only every third month.  Even if people


Consider first the SPF forecasts.  On the other hand, m

Assuming that the professional forecasters whom the 
journalists interview for their stories are able to process the new 
inflation data quickly, it seems likely that the news stories on 
inflation in mid-month will reflect mostly the same forecasts that are 
later reported in the SPF.

However, by the time month $t+2$ ends the SPF forecast is more than 
two and a half months out of date.  There will have been two new 
statistical reports about both inflation and unemployment, income, 
retail sales, and capacity utilization that have arrived.  Assuming 
that news reports through the end of month $t+2$ continue to reflect 
the same forecast that was collected in the middle of month $t$ seems 
inappropriate.  The situation is even worse for the first half of 
month $t+3$, because there will have been three reports on 
unemployment, income, and retail sales since the last SPF forecasts 
were collected.  The safest course would seem to be to estimate
\eqref{eq:monthlym3} only on data from months that were the final
month in a quarter, where the model can most realistically be said
to apply.


One way to investigate this is through a simulation analysis in which
the process for $N_{t}$ is taken to be the actual time path of SPF
forecasts, and we assume that $\lambda_{1}= 0.15$ while $\lambda_{2} =
0.35$, yielding an {\it average} absorption rate of 0.25.  Results of
estimating~\eqref{eq:base} on this population are presented in
Table~\ref{table:estonsim}.  Fortunately, the point estimate is quite
close to 0.25, which suggests that when there is heterogeneity in
$\lambda$, estimation of our baseline model will return a $\lambda$
that is close to the mean $\lambda$ in the population.

The final row of the table presents results from an intermediate case 
where the $\lambda$'s are heterogeneous within a restricted range from 
$\lambda = 0.2$ to $\lambda = 0.3$.  Again the estimate of $\lambda$ 
is very close to its population mean, but the Durbin-Watson statistic 
indicates that correlation consequences of $\lambda$ heterogeneity are 
much more modest.


Inflation 
expectations were chosen as the primary modeling target for two 
reasons.  First, standard theories of inflation imply that workers' 
inflation expectations can importantly influence inflation outcomes: 
Workers who expect higher inflation should push for greater wage 
increases, pushing up employers' costs and therefore prices.  Second, 
inflation expectations are one of the few macroeconomic variables for 
which a long time series exists on the expectations of the typical 
household.  The expectations data come from the University of 
Michigan's monthly Survey of Consumers, which is also the source for 
the closely-watched consumer sentiment index.  The baseline version of 
the model generates a simple analytical expression for the evolution 
of expectations that can be estimated using the published Michigan 
data.  The model turns out to do a remarkably good job in explaining 
these data, yielding an estimate that at any given time only about a 
quarter of households have fully up-to-date inflation expectations.


If not, the most fruitful direction to explore 
next would probably be whether the literature on `small world' social 
networks could be applied to the problem.  (The small world literature 
examines the patterns of connection and transmission of information 
among members of social networks that exhibit both local 
(neighborhood) and global connections.  See 
Watts~\cite{watts:smallworld} for an excellent overview.)

The primary conclusion is that the baseline equation for the model 
without social communication does a very good job in capturing the 
aggregate dynamics of inflation expectations even when social 
communication exists.  However, the estimate of the 
speed-of-transmission parameter $\lambda$ is biased upward, because 
news now spreads faster through the population than in the baseline 
model.


Intutitively it might seem that if almost 70 percent of agents have 
inflation expectations that are of a vintage of a year or less, the 
behavior of the macroeconomy could not be all that different from what 
would be expected if all expectations were completely up-to-date.  The 
surprising message of 
Roberts~\cite{roberts:stickyinfl,roberts:phillips} and Mankiw and 
Reis~\cite{mankiw&reis:slumps,mankiw&reis:stickye} is that this intuition is wrong.  
Mankiw and Reis show that an economy with $\lambda=0.25$ behaves in 
ways that are sharply different from an economy with fully rational 
expectations ($\lambda=1$), and argue that in each case where behavior 
is different the behavior of the $\lambda=0.25$ economy corresponds 
better with empirical evidence (for example with respect to the effect 
of interest rate shocks on aggregate output).

Mankiw and Reis~\cite{mankiw&reis:slumps,mankiw&reis:stickye} simply postulated 
$\lambda=0.25$.  

Typical rational expectations models assume that all economic 
actors have full access to all macroeconomic data, agree upon the 
correct macroeconomic model, and use that model to update their 
expectations instantly when new macro statistics are released.  
More recently, critics have argued that various
features of macroeconomic data are inconsistent with rational
expectations (in particular, the high persistence of inflation (Fuhrer
and Moore~\cite{fuhrer&moore:persistence}) and the apparent
inexorability of the tradeoff between inflation and unemployment) are
inconsistent with rational expectations (Ball~\cite{ball:sacrifice};
Mankiw~\cite{mankiw:inexorable}).  The literature has begun to respond
by taking up Friedman's challenge to explicitly model how agents
formulate expectations; Sargent~\cite{sargent:bounded}, for example,
explores models in which agents learn by performing regressions on
available macroeconomic data, while Evans and
Honkapohja~\cite{evans&honk:book} survey a variety of alternative
models of macroeconomic learning.


As a recent survey by McCallum~\cite{mccallum:review} notes,
alternative models of expectations are generally tested by examining
whether they produce more plausible macroeconomic dynamics than
standard rational expectations models.  
